\documentclass[12pt]{article}
\usepackage{amsmath}
\usepackage{amsfonts}
\usepackage{amssymb}
\usepackage{graphicx}
\usepackage{tikz}
\usepackage{array}
\usepackage{booktabs}
\usepackage{hyperref}
\usepackage{geometry}
\geometry{margin=2.5cm}
\usepackage{xepersian}
\settextfont[Path="./", Extension=".ttf"]{XB-Niloofar}
\setdigitfont[Path="./", Extension=".ttf"]{XB-Niloofar}
\begin{document}
	
	\title{سناریو ۴: رابطه بین یک متغیر EEG پیوسته و یک نمره رفتاری پیوسته (همبستگی)}
	\author{محمدمهدی شریف بیگی}
	\maketitle
	
	\section{مثال: همبستگی در مطالعه پارکینسون}
	
	\subsection{زمینه پژوهش}
	مطالعه بر روی پتانسیل‌های برانگیخته شنوایی P300 در بیماران پارکینسون و ارتباط آن با عملکرد شناختی، بررسی می‌کند که آیا بین دامنه P300 و نمرات ارزیابی شناختی مونترال (MoCA) رابطه‌ای وجود دارد یا خیر.
	
	\subsection{سوال پژوهشی}
	آیا بین دامنه پتانسیل برانگیخته P300 و عملکرد حافظه کاری در بیماران پارکینسون رابطه‌ای وجود دارد؟
	
	\subsection{فرضیه‌ها}
	
	\textbf{فرضیه صفر ($H_0$):} هیچ رابطه خطی بین دامنه P300 و دقت حافظه کاری وجود ندارد ($\rho = 0$).\\
	\textbf{فرضیه جایگزین ($H_1$):} رابطه خطی بین دامنه P300 و دقت حافظه کاری وجود دارد ($\rho \neq 0$).
	
	\subsection{ساختار داده‌ها}
	
	\textbf{متغیرها:} دو متغیر پیوسته که برای هر آزمودنی اندازه‌گیری شده است.
	
	\textbf{متغیر ۱:} دامنه P300 (میکروولت) در الکترود Pz
	\begin{itemize}
		\item نوع داده: پیوسته
		\item واحد اندازه‌گیری: میکروولت (\lr{$\mu$V})
		\item محدوده معمول: ۲ تا ۲۰ میکروولت
		\item الکترود ثبت: Pz (ناحیه آهیانه‌ای)
	\end{itemize}
	
	\textbf{متغیر ۲:} دقت حافظه کاری (درصد پاسخ‌های صحیح)
	\begin{itemize}
		\item نوع داده: پیوسته
		\item واحد اندازه‌گیری: درصد
		\item محدوده: ۰ تا ۱۰۰ درصد
		\item ابزار ارزیابی: آزمون n-back
	\end{itemize}
	
	\subsubsection{مفهوم پتانسیل برانگیخته P300}
	
	\textbf{P300 چیست؟}
	\begin{itemize}
		\item پتانسیل برانگیخته درون‌زا (\lr{Endogenous ERP})
		\item ظاهر می‌شود حدود ۳۰۰ میلی‌ثانیه بعد از محرک
		\item مرتبط با فرآیندهای شناختی: توجه، تصمیم‌گیری، حافظه کاری
		\item در پارادایم oddball اندازه‌گیری می‌شود
	\end{itemize}
	
	\textbf{مولفه‌های P300:}
	\begin{itemize}
		\item \textbf{دامنه (Amplitude):} شدت فعالیت عصبی (میکروولت)
		\item \textbf{تأخیر (Latency):} سرعت پردازش اطلاعات (میلی‌ثانیه)
		\item \textbf{توپولوژی (Topography):} توزیع فضایی روی پوست سر
	\end{itemize}
	
	\textbf{اهمیت در پارکینسون:}
	\begin{itemize}
		\item شاخص عینی برای ارزیابی عملکرد شناختی
		\item حساس به تغییرات زودهنگام شناختی
		\item مستقل از علائم حرکتی بیماری
	\end{itemize}
	
	\begin{figure}[h]
		\centering
		\begin{tikzpicture}[scale=0.8]
			% Time axis
			\draw[thick, ->] (0,0) -- (8,0) node[right] {زمان (ms)};
			\foreach \x/\label in {0/0, 2/100, 4/200, 6/300, 8/400}
			\draw (\x,0.1) -- (\x,-0.1) node[below] {\label};
			
			% Voltage axis  
			\draw[thick, ->] (0,-2) -- (0,3) node[above] {ولتاژ (μV)};
			\foreach \y/\label in {-2/-10, -1/-5, 0/0, 1/5, 2/10}
			\draw (0.1,\y) -- (-0.1,\y) node[left] {\label};
			
			% ERP waveform
			\draw[thick, blue] (0,0) .. controls (1,-0.5) and (1.5,-1) .. (2,-1.2)
			.. controls (2.5,-1) and (3,-0.5) .. (4,0.5)
			.. controls (5,1.5) and (5.5,2) .. (6,2.2)
			.. controls (6.5,1.8) and (7,1) .. (8,0.2);
			
			% P300 peak annotation
			\draw[red, dashed] (6,0) -- (6,2.2);
			\node[red, above] at (6,2.5) {P300};
			\node[red] at (6.5,1.5) {\rl{دامنه}: ۱۰ $\mu$V};
			\node[red] at (6.5,1) {\rl{تأخیر}: ۳۰۰ ms};
			
			% N1, P2 annotations
			\node[gray] at (2,-1.5) {N1};
			\node[gray] at (4,0.8) {P2};
			
		\end{tikzpicture}
		\caption{نمونه موج P300 در پاسخ به محرک هدف}
	\end{figure}
	
	\subsection{آزمون آماری توصیه شده}
	
	\textbf{همبستگی پیرسون} (اگر داده‌ها نرمال باشند) یا \textbf{همبستگی اسپیرمن} (برای داده‌های غیرنرمال یا ترتیبی).
	
	\subsubsection{انتخاب نوع همبستگی}
	
	\begin{table}[h]
		\centering
		\caption{معیارهای انتخاب نوع همبستگی}
		\begin{tabular}{|p{4cm}|p{5cm}|p{5cm}|}
			\hline
			\textbf{ویژگی} & \textbf{همبستگی پیرسون} & \textbf{همبستگی اسپیرمن} \\
			\hline
			نوع داده & متغیرهای کمّی پیوسته & متغیرهای ترتیبی یا کمّی \\
			\hline
			توزیع داده & نرمال (یا نزدیک به نرمال) & هر نوع توزیعی \\
			\hline
			نوع رابطه & خطی & یکنواخت (خطی یا غیرخطی) \\
			\hline
			حساسیت به نقاط پرت & حساس & مقاوم \\
			\hline
			قدرت آماری & بیشتر (اگر فرضیات برقرار باشند) & کمتر اما مقاوم‌تر \\
			\hline
		\end{tabular}
	\end{table}
	
	\textbf{در مطالعه پارکینسون:}
	\begin{itemize}
		\item داده‌های EEG اغلب کج توزیع هستند
		\item حجم نمونه معمولاً کوچک است
		\item احتمال وجود نقاط پرت بالاست
		\item \textbf{توصیه:} همبستگی اسپیرمن
	\end{itemize}
	
	\subsection{روش اجرای مطالعه}
	
	\subsubsection{شرکت‌کنندگان}
	\begin{itemize}
		\item ۳۲ بیمار مبتلا به پارکینسون ایدیوپاتیک
		\item محدوده سنی: ۴۵-۷۵ سال
		\item معیارهای ورود: تشخیص قطعی پارکینسون، نمره MoCA بالای ۱۸
		\item معیارهای خروج: بیماری‌های عصبی همراه، مشکلات شنوایی شدید
	\end{itemize}
	
	\subsubsection{ابزارها و روش‌ها}
	
	\textbf{ثبت EEG:}
	\begin{itemize}
		\item سیستم ۶۴ کاناله بر اساس استاندارد ۱۰-۲۰
		\item نرخ نمونه‌برداری: ۵۱۲ هرتز
		\item فیلتر: ۰.۱-۳۰ هرتز
		\item الکترود مرجع: mastoids
		\item الکترود زمین: پیشانی
	\end{itemize}
	
	\textbf{پارادایم Oddball شنوایی:}
	\begin{itemize}
		\item محرک استاندارد: تون ۱۰۰۰ هرتز (احتمال ۸۰٪)
		\item محرک هدف: تون ۲۰۰۰ هرتز (احتمال ۲۰٪)
		\item مدت محرک: ۵۰ میلی‌ثانیه
		\item فاصله بین محرک‌ها: ۱-۲ ثانیه (تصادفی)
		\item تعداد کل محرک‌ها: ۲۰۰ (۴۰ هدف، ۱۶۰ استاندارد)
	\end{itemize}
	
	\textbf{آزمون حافظه کاری (n-back):}
	\begin{itemize}
		\item محرک‌های بصری: حروف الفبا
		\item مدت ارائه: ۵۰۰ میلی‌ثانیه
		\item فاصله بین محرک‌ها: ۲۰۰۰ میلی‌ثانیه
		\item سطوح: ۱-back، ۲-back، ۳-back
		\item تعداد کوشش در هر سطح: ۲۰
	\end{itemize}
	
\begin{figure}[h]
	\centering
	\begin{tikzpicture}[scale=1.2]
		% Main timeline
		\draw[thick, ->] (0,0) -- (14,0) node[right] {زمان};
		
		% Oddball paradigm title
		\node[anchor=west, font=\large] at (0,4) {محرک‌های Oddball};
		
		% Oddball stimuli - drawing each one individually
		\draw[blue, very thick] (1.5,3) -- (1.5,2.5);
		\node[blue, font=\small] at (1.5,2.1) {S};
		\draw[blue, very thick] (2.5,3) -- (2.5,2.5);
		\node[blue, font=\small] at (2.5,2.1) {S};
		\draw[red, very thick] (3.5,3) -- (3.5,2.5);
		\node[red, font=\small] at (3.5,2.1) {T};
		\draw[blue, very thick] (4.5,3) -- (4.5,2.5);
		\node[blue, font=\small] at (4.5,2.1) {S};
		\draw[blue, very thick] (5.5,3) -- (5.5,2.5);
		\node[blue, font=\small] at (5.5,2.1) {S};
		\draw[red, very thick] (6.5,3) -- (6.5,2.5);
		\node[red, font=\small] at (6.5,2.1) {T};
		\draw[blue, very thick] (7.5,3) -- (7.5,2.5);
		\node[blue, font=\small] at (7.5,2.1) {S};
		\draw[blue, very thick] (8.5,3) -- (8.5,2.5);
		\node[blue, font=\small] at (8.5,2.1) {S};
		\draw[blue, very thick] (9.5,3) -- (9.5,2.5);
		\node[blue, font=\small] at (9.5,2.1) {S};
		\draw[red, very thick] (10.5,3) -- (10.5,2.5);
		\node[red, font=\small] at (10.5,2.1) {T};
		
		% Oddball legend with proper spacing
		\node[blue, anchor=west, font=\footnotesize] at (0,1.5) {\lr{S}: استاندارد 1000 Hz};
		\node[red, anchor=west, font=\footnotesize] at (7,1.5) {\lr{T}: \rl{هدف} ۲۰۰۰ Hz};
		
		% Separator line
		\draw[gray, dashed] (0,0.8) -- (12,0.8);
		
		% N-back task title
		\node[anchor=west, font=\large] at (0,0.3) {آزمون n-back};
		
		% N-back letters with more spacing
		\foreach \x/\letter in {1.5/A, 2.5/B, 3.5/A, 4.5/C, 5.5/B, 6.5/B, 7.5/D, 8.5/A, 9.5/D, 10.5/D}
		{
			\draw[green!60!black, thick] (\x,-0.5) rectangle (\x+0.6,-1);
			\node[green!60!black, font=\small] at (\x+0.3,-0.75) {\letter};
		}
		
		% N-back explanation
		\node[green!60!black, anchor=west, font=\tiny] at (0,-1.8) {مثال\rl{:} تشخیص ۲-\lr{back}};
		\node[green!60!black, anchor=west, font=\tiny] at (0,-2.1) {\rl{حرف A در موقعیت ۳} = \rl{موقعیت ۱}};
		
	\end{tikzpicture}
	\caption{نمایش شماتیک پروتکل آزمایش}
\end{figure}
	
	\subsection{تجزیه و تحلیل داده‌ها}
	
	\subsubsection{پردازش داده‌های EEG}
	
	\textbf{پیش‌پردازش:}
	\begin{itemize}
		\item حذف artifacts (چشم، ماهیچه، حرکت)
		\item فیلتر کردن: \lr{bandpass 0.1-30 Hz}
		\item بازنمونه‌برداری به ۲۵۶ هرتز
		\item \lr{ICA} برای حذف \lr{artifacts} باقی‌مانده
	\end{itemize}
	
	\textbf{استخراج P300:}
	\begin{itemize}
		\item میانگین‌گیری epochs مربوط به محرک‌های هدف
		\item پنجره زمانی: \lr{-200} تا \lr{+800} میلی‌ثانیه
		\item \lr{ baseline correction}: \lr{-200} تا ۰ میلی‌ثانیه
		\item اندازه‌گیری دامنه: حداکثر مثبت در پنجره \lr{250}-\lr{500} میلی‌ثانیه
		\item الکترود تحلیل: Pz، Cz، Fz
	\end{itemize}
	
	\subsubsection{محاسبه نمره حافظه کاری}
	
	\textbf{شاخص‌های عملکرد:}
	\begin{itemize}
		\item دقت: درصد پاسخ‌های صحیح
		\item زمان واکنش: میانگین زمان پاسخ به محرک‌های هدف
		\item شاخص حساسیت (\lr{d-prime}): $d' = Z(\text{Hit Rate}) - Z(\text{False Alarm Rate})$
		\item شاخص تمایل پاسخ ($\beta$): معیار محافظه‌کاری در پاسخ‌دهی
	\end{itemize}
	
	\textbf{نمره ترکیبی حافظه کاری:}
	$$\text{WM Score} = \frac{(d'_{1back} + d'_{2back} + d'_{3back})}{3}$$
	
	\subsection{روش‌های آماری}
	
	\subsubsection{آمار توصیفی}
	\begin{itemize}
		\item میانگین و انحراف معیار برای هر متغیر
		\item بررسی نرمال بودن با آزمون Shapiro-Wilk
		\item شناسایی نقاط پرت با روش Z-score (|Z| > 3.29)
		\item نمودارهای پراکندگی برای بررسی بصری رابطه
	\end{itemize}
	
	\subsubsection{تحلیل همبستگی}
	
	\textbf{آزمون اصلی:}
	$$H_0: \rho = 0 \quad \text{در مقابل} \quad H_1: \rho \neq 0$$
	
	\textbf{ضریب همبستگی اسپیرمن:}
	$$r_s = 1 - \frac{6\sum d_i^2}{n(n^2-1)}$$
	
	\textbf{آزمون معناداری:}
	$$t = r_s \sqrt{\frac{n-2}{1-r_s^2}}$$
	
	درجه آزادی: $df = n-2$
	
	\subsubsection{تفسیر اندازه اثر (\lr{Effect Size})}
	
	\begin{table}[h]
		\centering
		\caption{راهنمای تفسیر ضریب همبستگی}
		\begin{tabular}{|p{3cm}|p{3cm}|p{6cm}|}
			\hline
			\textbf{مقدار |r|} & \textbf{اندازه اثر} & \textbf{تفسیر} \\
			\hline
			0.10 - 0.29 & کوچک & رابطه ضعیف \\
			\hline
			0.30 - 0.49 & متوسط & رابطه متوسط \\
			\hline
			0.50 - 0.69 & بزرگ & رابطه قوی \\
			\hline
			0.70 - 0.89 & خیلی بزرگ & رابطه خیلی قوی \\
			\hline
			0.90+ & تقریباً کامل & رابطه تقریباً کامل \\
			\hline
		\end{tabular}
	\end{table}
	
	\section{نتایج مطالعه}
	
	\subsection{آمار توصیفی}
	
	\begin{table}[h]
		\centering
		\caption{آمار توصیفی متغیرهای اصلی}
		\begin{tabular}{lcccc}
			\toprule
			\textbf{متغیر} & \textbf{میانگین} & \textbf{انحراف معیار} & \textbf{دامنه} & \textbf{میانه} \\
			\midrule
			دامنه P300 (μV) & 8.4 & 3.2 & 3.1 - 16.8 & 7.9 \\
			نمره حافظه کاری (%) & 76.3 & 12.8 & 45 - 95 & 78.5 \\
			MoCA کل & 24.1 & 3.4 & 18 - 30 & 25 \\
			\bottomrule
		\end{tabular}
	\end{table}
	
	\subsection{تحلیل همبستگی}
	
	\textbf{فرضیه‌ها:}
	\begin{itemize}
		\item $H_0$: هیچ رابطه خطی بین دامنه P300 و عملکرد حافظه کاری وجود ندارد ($\rho = 0$)
		\item $H_1$: رابطه خطی بین دامنه P300 و عملکرد حافظه کاری وجود دارد ($\rho \neq 0$)
	\end{itemize}
	
	\textbf{آزمون آماری:} ضریب همبستگی اسپیرمن
	
	\textbf{نتایج:}
	\begin{itemize}
		\item ضریب همبستگی: $r_s = +0.65$
		\item $p$-value < 0.001
		\item $R^2 = 0.42$
		\item فاصله اطمینان ۹۵٪: [0.38, 0.82]
	\end{itemize}
	
	\begin{figure}[h]
		\centering
		\begin{tikzpicture}[scale=0.8]
			% Axes
			\draw[thick, ->] (0,0) -- (10,0) node[right] {دامنه P300 (μV)};
			\draw[thick, ->] (0,0) -- (0,8) node[above] {نمره حافظه کاری (\%)};
			
			% Scale
			\foreach \x/\label in {0/0, 2/5, 4/10, 6/15, 8/20}
			\draw (\x,0.1) -- (\x,-0.1) node[below] {\label};
			
			\foreach \y/\label in {0/40, 2/60, 4/80, 6/100}
			\draw (0.1,\y) -- (-0.1,\y) node[left] {\label};
			
			% Data points (simulated)
			\foreach \x/\y in {1.5/1.2, 2.1/1.8, 2.8/2.1, 3.2/2.5, 3.8/2.9, 4.1/3.2, 4.6/3.6, 5.2/4.1, 5.8/4.5, 6.1/4.8, 6.7/5.2, 7.3/5.6, 7.9/6.1}
			{
				\fill[blue] (\x,\y) circle (0.08);
			}
			
			% Regression line
			\draw[red, thick] (1,1) -- (8,6.5);
			
			% Correlation info
			\node[red] at (7,2) {$r_s = +0.65$};
			\node[red] at (7,1.5) {$p < 0.001$};
			\node[red] at (7,1) {$R^2 = 0.42$};
			
		\end{tikzpicture}
		\caption{نمودار پراکندگی: رابطه بین دامنه P300 و عملکرد حافظه کاری}
	\end{figure}
	
	\subsection{تفسیر نتایج}
	
	\textbf{یافته اصلی:}
	فرضیه صفر رد شد. همبستگی مثبت قوی و معناداری بین دامنه P300 و عملکرد حافظه کاری وجود دارد.
	
	\textbf{تفسیر علمی:}
	\begin{itemize}
		\item بیمارانی که دامنه P300 بیشتری داشتند، عملکرد بهتری در آزمون‌های حافظه کاری نشان دادند
		\item دامنه P300 ۴۲٪ از تغییرات عملکرد حافظه کاری را تبیین می‌کند
		\item این یافته نشان می‌دهد P300 شاخص عینی مناسبی برای ارزیابی توانایی‌های شناختی در پارکینسون است
	\end{itemize}
	
	\textbf{اهمیت بالینی:}
	\begin{itemize}
		\item P300 می‌تواند به عنوان biomarker برای تشخیص زودهنگام اختلالات شناختی استفاده شود
		\item این روش مستقل از علائم حرکتی است و تحت تأثیر داروهای ضدپارکینسون قرار نمی‌گیرد
		\item امکان پیگیری روند تغییرات شناختی در طول زمان را فراهم می‌کند
	\end{itemize}
	
	\section{تحلیل‌های تکمیلی}
	
	\subsection{همبستگی جزئی (\lr{Partial Correlation})}
	
	برای کنترل متغیرهای مخدوشگر احتمالی:
	
	\textbf{متغیرهای کنترل:}
	\begin{itemize}
		\item سن
		\item تحصیلات
		\item مدت بیماری
		\item شدت علائم حرکتی (UPDRS-III)
	\end{itemize}
	
	\textbf{نتیجه:}
	$$r_{partial} = +0.58, \quad p < 0.01$$
	
	رابطه همچنان معنادار و قوی باقی ماند.
	
	\subsection{تحلیل رگرسیون}
	
	برای پیش‌بینی عملکرد حافظه کاری:
	
	\textbf{مدل رگرسیون:}
	$$\text{WM Score} = \beta_0 + \beta_1 \times \text{P300 Amplitude} + \varepsilon$$
	
	\textbf{نتایج:}
	\begin{itemize}
		\item $\beta_0 = 45.2$ (عرض از مبدا)
		\item $\beta_1 = 3.7$ (شیب)
		\item $R^2 = 0.42$
		\item $F(1,30) = 21.8, p < 0.001$
	\end{itemize}
	
	\textbf{تفسیر:} به ازای هر میکروولت افزایش در دامنه P300، نمره حافظه کاری ۳.۷ درصد افزایش می‌یابد.
	
	\section{محدودیت‌ها و ملاحظات}
	
	\subsection{محدودیت‌های روش‌شناختی}
	
	\begin{itemize}
		\item \textbf{علیت:} همبستگی علیت را اثبات نمی‌کند
		\item \textbf{حجم نمونه:} نمونه نسبتاً کوچک (n=32)
		\item \textbf{طرح مقطعی:} امکان بررسی تغییرات طولی وجود ندارد
		\item \textbf{متغیرهای مخدوشگر:} ممکن است متغیرهای پنهان تأثیرگذار باشند
	\end{itemize}
	
	\subsection{ملاحظات آماری}
	
	\begin{itemize}
		\item \textbf{فرض خطی بودن:} رابطه ممکن است غیرخطی باشد
		\item \textbf{نقاط پرت:} حساسیت به نقاط پرت احتمالی
		\item \textbf{توزیع داده‌ها:} ضرورت بررسی نرمال بودن
		\item \textbf{اندازه اثر:} تأکید بر اهمیت عملی علاوه بر معناداری آماری
	\end{itemize}
	
	\subsection{پیشنهادات برای مطالعات آینده}
	
	\begin{itemize}
		\item \textbf{مطالعه طولی:} پیگیری بیماران در طول زمان
		\item \textbf{حجم نمونه بیشتر:} افزایش قدرت آماری
		\item \textbf{گروه کنترل:} مقایسه با افراد سالم همسن
		\item \textbf{متغیرهای اضافی:} بررسی سایر پارامترهای EEG
		\item \textbf{تکرارپذیری:} اعتبارسنجی در نمونه‌های مستقل
	\end{itemize}
	
	\section{نتیجه‌گیری}
	
	این مطالعه نمونه عالی از \textbf{تحلیل همبستگی} در تحقیقات عصب‌روانشناسی است که در آن:
	
	\begin{itemize}
		\item دو متغیر پیوسته (دامنه P300 و عملکرد حافظه کاری) بررسی شدند
		\item از همبستگی اسپیرمن برای تحلیل رابطه استفاده شد
		\item رابطه مثبت قوی و معناداری یافت شد
		\item یافته‌ها اهمیت بالینی قابل توجهی دارند
	\end{itemize}
	
	\textbf{کاربردهای بالینی:}
	\begin{itemize}
		\item P300 به عنوان biomarker غیرتهاجمی برای ارزیابی شناختی
		\item تشخیص زودهنگام اختلالات شناختی در پارکینسون
		\item پیگیری اثربخشی مداخلات درمانی شناختی
		\item ارزیابی عینی مستقل از علائم حرکتی
	\end{itemize}
	
	\textbf{اهمیت روش‌شناختی:}
	\begin{itemize}
		\item نمونه کاربرد صحیح تحلیل همبستگی
		\item تأکید بر اهمیت کنترل متغیرهای مخدوشگر
		\item نشان‌دهنده ضرورت تفسیر محتاطانه یافته‌های همبستگی
		\item پایه‌ای برای طراحی مطالعات علّی آینده
	\end{itemize}
	
	\section{فرمول‌های آماری}
	
	\subsection{ضریب همبستگی اسپیرمن}
	
	$r_s = 1 - \frac{6\sum_{i=1}^{n} d_i^2}{n(n^2-1)}$
	
	که در آن:
	\begin{itemize}
		\item $d_i$: تفاوت رتبه‌ها برای مشاهده $i$ام
		\item $n$: تعداد مشاهدات
	\end{itemize}
	
	\subsection{آزمون معناداری}
	
	$t = r_s \sqrt{\frac{n-2}{1-r_s^2}}$
	
	با درجه آزادی: $df = n-2$
	
	\subsection{فاصله اطمینان}
	
	برای محاسبه فاصله اطمینان ۹۵٪ برای ضریب همبستگی:
	
	$\text{CI} = \tanh\left(\tanh^{-1}(r) \pm \frac{1.96}{\sqrt{n-3}}\right)$
	
	\subsection{ضریب تعیین}
	
	$R^2 = r_s^2$
	
	که درصد واریانس مشترک بین دو متغیر را نشان می‌دهد.
	
	\section{مراجع}
	
	\begin{latin}
		\begin{enumerate}
			\item Kumar, A., Singh, V., Gupta, S., et al. (2023). Auditory evoked P300 potential in patients with Parkinson's disease. \textit{Cureus}, 15(9), e45127. \\
			Available at: \url{https://www.ncbi.nlm.nih.gov/pmc/articles/PMC10599456/}
		\end{enumerate}
	\end{latin}
	
\end{document}
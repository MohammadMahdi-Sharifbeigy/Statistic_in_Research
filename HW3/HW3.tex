\documentclass[12pt]{article}
% Load packages BEFORE xepersian
\usepackage{geometry}
\usepackage{xcolor}
\usepackage{amsmath}
\usepackage{amsfonts}
\usepackage{graphicx}
\usepackage{pgfplots}
\pgfplotsset{compat=1.18}
\usepackage{booktabs}
\usepackage{longtable}
\usepackage{array}
\usepackage{fancyhdr}
\usepackage{titlesec}
\usepackage{tcolorbox}
\usepackage{hyperref}
% Listings package with custom setup
\usepackage{listings}
% XePersian setup for RTL/LTR support - MUST BE LOADED LAST
\usepackage{xepersian}
\settextfont[Path="./", Extension=".ttf"]{XB-Niloofar}
\setdigitfont[Path="./", Extension=".ttf"]{XB-Niloofar}

% Page setup
\geometry{
	top=2.5cm,
	bottom=2.5cm,
	left=2.5cm,
	right=2.5cm
}
\setlength{\headheight}{16pt}

% Colors
\definecolor{codegreen}{rgb}{0,0.6,0}
\definecolor{codegray}{rgb}{0.5,0.5,0.5}
\definecolor{codepurple}{rgb}{0.58,0,0.82}
\definecolor{backcolour}{rgb}{0.95,0.95,0.92}
\definecolor{sectioncolor}{rgb}{0.1,0.4,0.7}
\definecolor{examplecolor}{rgb}{0.9, 0.95, 1.0}
\definecolor{anovacolor}{rgb}{0.95, 0.9, 1.0}

% Header and footer
\pagestyle{fancy}
\fancyhf{}
\fancyhead[R]{\lr{Outliers in Laplace Distribution}}
\fancyhead[L]{Outliers در توزیع لاپلاس}
\fancyfoot[C]{\thepage}

% Section styling
\titleformat{\section}
{\color{sectioncolor}\Large\bfseries}
{\thesection}{1em}{}
\titleformat{\subsection}
{\color{sectioncolor}\large\bfseries}
{\thesubsection}{1em}{}

% Hyperlink setup
\hypersetup{
	colorlinks=true,
	linkcolor=blue,
	filecolor=magenta,      
	urlcolor=cyan,
	pdfpagemode=FullScreen,
	unicode=true,
	pdfencoding=auto
}

\title{
	\begin{center}
		{\Huge \bfseries موقعیت Outliers در توزیع لاپلاس} \\
		\vspace{1cm}
		{\large بررسی مقایسه‌ای با توزیع نرمال}
	\end{center}
}
\author{تحلیل آماری}
\date{\today}

\begin{document}
	
	\maketitle
	
	\section{مقدمه}
	
	سوال اصلی این است که \textbf{outlierها در توزیع لاپلاسی در کجا قرار دارند و چرا؟}
	
	\begin{tcolorbox}[colback=examplecolor, colframe=sectioncolor, title=پاسخ کلی]
		در توزیع لاپلاسی، outlierها کمتر مشهود هستند و در واقع این توزیع outlierهای کمتری نسبت به توزیع نرمال تولید می‌کند.
	\end{tcolorbox}
	
	\section{موقعیت Outliers در توزیع لاپلاسی}
	
	\subsection{مقایسه با توزیع نرمال}
	
	\begin{figure}[htbp]
		\centering
		\includegraphics[width=0.9\textwidth]{laplace_outliers_comparison.png}
		\caption{مقایسه موقعیت \lr{outliers} در توزیع نرمال و لاپلاس. نقاط آبی (\lr{outliers} نرمال) نزدیک‌تر به مرکز و نقاط قرمز (\lr{outliers} لاپلاس) دورتر از مرکز قرار دارند. این نمودار نشان می‌دهد که توزیع لاپلاس (خط آبی چین‌دار) دارای دم‌های سنگین‌تر نسبت به توزیع نرمال (خط زرد) است.}
		\label{fig:outliers_comparison}
	\end{figure}
	
	\begin{itemize}
		\item \textbf{نقاط آبی (\lr{Outliers Normal})}: در مناطق نزدیک‌تری به مرکز قرار دارند (حدود $x \approx \pm 4$)
		\item \textbf{نقاط قرمز (\lr{Outliers Laplace})}: در مناطق دورتری از مرکز قرار دارند (حدود $x \approx \pm 7$ یا $\pm 8$)
	\end{itemize}
	
	\begin{tcolorbox}[colback=examplecolor, colframe=sectioncolor, title=مشاهدات کلیدی از نمودار]
		\begin{itemize}
			\item \textbf{توزیع نرمال (خط زرد)}: دم‌های نازک‌تر دارد، بنابراین مقادیری که در $x \approx \pm 4$ قرار دارند به سرعت به outlier تبدیل می‌شوند
			\item \textbf{توزیع لاپلاس (خط آبی چین‌دار)}: دم‌های پهن‌تر دارد، بنابراین برای اینکه یک نقطه outlier محسوب شود باید خیلی دورتر از مرکز باشد
		\end{itemize}
	\end{tcolorbox}
	
	\newpage
	
	\section{دلایل این تفاوت}
	
	\subsection{ Heavy Tails (دم‌های سنگین)}
	
	\begin{tcolorbox}[colback=anovacolor, colframe=sectioncolor]
		توزیع لاپلاسی دارای دم‌های سنگین‌تری نسبت به توزیع نرمال است. این یعنی:
		\begin{itemize}
			\item مقادیر دورتر از مرکز احتمال بیشتری دارند که رخ دهند
			\item آنچه در توزیع نرمال outlier محسوب می‌شود، در لاپلاسی ممکن است طبیعی باشد
		\end{itemize}
	\end{tcolorbox}
	
	\subsection{شکل توزیع}
	
	توزیع لاپلاسی دارای ویژگی‌های زیر است:
	
	\begin{itemize}
		\item \textbf{قله تیزتر (Sharper Peak)}: نسبت به توزیع نرمال
		\item \textbf{دم‌های آهسته‌تر}: به آرامی کاهش می‌یابند
		\item \textbf{ساختار مقاوم}: مقادیر افراطی بیشتر در محدوده قابل قبول قرار می‌گیرند
	\end{itemize}
	
	\subsection{تحلیل ریاضی}
	
	تابع چگالی احتمال توزیع‌ها:
	
	\textbf{توزیع نرمال:}
	$$f(x) = \frac{1}{\sigma\sqrt{2\pi}} e^{-\frac{(x-\mu)^2}{2\sigma^2}}$$
	
	\textbf{توزیع لاپلاس:}
	$$f(x) = \frac{1}{2b} e^{-\frac{|x-\mu|}{b}}$$
	
	تفاوت کلیدی در نحوه کاهش تابع در دم‌هاست:
	\begin{itemize}
		\item نرمال: کاهش نمایی با توان دوم ($e^{-x^2}$)
		\item لاپلاس: کاهش نمایی خطی ($e^{-|x|}$)
	\end{itemize}
	
	\section{کاربردهای عملی}
	
	\subsection{کاربردهای توزیع لاپلاسی}
	
	توزیع لاپلاسی به دلیل مقاومت در برابر outlierها در موارد زیر استفاده می‌شود:
	
	\begin{enumerate}
		\item \textbf{Robust Statistics}: وقتی می‌خواهیم نسبت به outlierها مقاوم باشیم
		\item \textbf{Lasso Regression}: در یادگیری ماشین برای تنظیم‌سازی
		\item \textbf{پردازش سیگنال}: جایی که نویز دارای outlier است
		\item \textbf{تحلیل تصاویر}: برای حذف نویزهای نقطه‌ای
		\item \textbf{آمار بیزی}: به عنوان prior مناسب
		\item \textbf{مدل‌سازی مالی}: برای بازارهای ناپایدار
	\end{enumerate}
	
	\section{نتیجه‌گیری}
	
	\begin{tcolorbox}[colback=examplecolor, colframe=sectioncolor, title=خلاصه پاسخ]
		\textbf{موقعیت outlierها در توزیع لاپلاسی:}
		\begin{itemize}
			\item در مناطق دورتری از مرکز نسبت به توزیع نرمال قرار می‌گیرند
			\item به دلیل heavy tails، کمتر به عنوان outlier شناخته می‌شوند
			\item این ویژگی باعث می‌شود لاپلاس برای داده‌های دارای نویز یا outlier طبیعی مناسب‌تر باشد
		\end{itemize}
	\end{tcolorbox}
	
	این تحلیل نشان می‌دهد که انتخاب توزیع مناسب بر اساس ماهیت داده‌ها و نحوه برخورد مطلوب با outlier ها بسیار مهم است.
	
\end{document}
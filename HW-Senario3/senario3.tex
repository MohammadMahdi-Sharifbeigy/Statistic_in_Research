\documentclass[12pt]{article}
\usepackage{amsmath}
\usepackage{amsfonts}
\usepackage{amssymb}
\usepackage{graphicx}
\usepackage{tikz}
\usepackage{array}
\usepackage{booktabs}
\usepackage{hyperref}
\usepackage{geometry}
\geometry{margin=2.5cm}
\usepackage{xepersian}
\settextfont[Path="./", Extension=".ttf"]{XB-Niloofar}
\setdigitfont[Path="./", Extension=".ttf"]{XB-Niloofar}
\begin{document}
	
	\title{سناریو ۳: دو گروه، دو یا چند نقطه زمانی/شرایط (طرح ترکیبی)}
	\author{محمدمهدی شریف بیگی}
	\maketitle
	
	\section{مثال: طرح ترکیبی در مطالعه پارکینسون}
	
	\subsection{زمینه پژوهش}
	مطالعه Subramanian و همکاران (۲۰۱۶) اثربخشی نوروفیدبک در بیماری پارکینسون را با استفاده از طرح ترکیبی بررسی کرده است.
	
	\subsection{فرضیه‌ها}
	
	\textbf{فرضیه صفر ($H_0$):} هیچ تعامل معناداری بین گروه و زمان وجود ندارد. تغییرات در نمرات حرکتی از پیش‌آزمون تا پس‌آزمون در هر دو گروه یکسان است.
	
	\textbf{فرضیه جایگزین ($H_1$):} تعامل معناداری بین گروه و زمان وجود دارد. تغییرات نمرات حرکتی در طول زمان به گروه‌بندی بیماران بستگی دارد.
	
	\subsection{ساختار داده‌ها}
	
	\textbf{عامل بین‌آزمودنی:} گروه (نوروفیدبک در مقابل کنترل)
	\begin{itemize}
		\item گروه ۱ (NF): نوروفیدبک + تمرین حرکتی (\lr{n=15})
		\item گروه ۲ (MOT): تمرین حرکتی تنها (\lr{n=15})
	\end{itemize}
	
	\textbf{عامل درون‌آزمودنی:} زمان (پیش‌تمرین در مقابل پس‌تمرین)
	
	\textbf{متغیر وابسته (\lr{DV}):} نمره مقیاس حرکتی MDS-UPDRS
	
	\subsubsection{مفاهیم بنیادی: عوامل بین‌آزمودنی و درون‌آزمودنی}
	
	\textbf{عامل بین‌آزمودنی (\lr{Between-Subjects Factor}):}
	\begin{itemize}
		\item هر شرکت‌کننده فقط در یک سطح از عامل قرار می‌گیرد
		\item شرکت‌کنندگان مختلف در سطوح مختلف قرار دارند
		\item مثال: گروه درمان (نوروفیدبک یا کنترل)
		\item فرد نمی‌تواند همزمان در هر دو گروه باشد
	\end{itemize}
	
	\textbf{عامل درون‌آزمودنی (\lr{Within-Subjects Factor}):}
	\begin{itemize}
		\item همان شرکت‌کننده در همه سطوح عامل قرار می‌گیرد
		\item اندازه‌گیری‌های مکرر از همان فرد
		\item مثال: زمان (پیش‌آزمون و پس‌آزمون)
		\item همان فرد در هر دو زمان اندازه‌گیری می‌شود
	\end{itemize}
	
	\begin{table}[h]
		\centering
		\caption{مقایسه عوامل بین‌آزمودنی و درون‌آزمودنی}
		\begin{tabular}{|p{3cm}|p{5cm}|p{5cm}|}
			\hline
			\textbf{ویژگی} & \textbf{بین‌آزمودنی} & \textbf{درون‌آزمودنی} \\
			\hline
			تخصیص افراد & هر فرد در یک سطح & هر فرد در همه سطوح \\
			\hline
			استقلال مشاهدات & مستقل & وابسته (همبسته) \\
			\hline
			کنترل تفاوت‌های فردی & ضعیف & قوی \\
			\hline
			تعداد شرکت‌کننده مورد نیاز & بیشتر & کمتر \\
			\hline
			خطر اثرات یادگیری/خستگی & ندارد & دارد \\
			\hline
			قدرت آماری & کمتر & بیشتر \\
			\hline
		\end{tabular}
	\end{table}
	
	\subsection{آزمون آماری توصیه شده}
	تحلیل واریانس ترکیبی (\lr{Mixed-Measures ANOVA}) - که شامل:
	\begin{itemize}
		\item عامل بین‌آزمودنی: گروه (نوروفیدبک در مقابل کنترل)
		\item عامل درون‌آزمودنی: زمان (پیش در مقابل پس)
	\end{itemize}
	
	\textbf{نکته مهم:} این آزمون ترکیبی است، نه صرفاً اندازه‌گیری‌های مکرر، زیرا:
	\begin{itemize}
		\item دارای هر دو نوع عامل است (بین‌آزمودنی + درون‌آزمودنی)
		\item هدف اصلی بررسی تعامل گروه × زمان است
		\item از \lr{Split-Plot ANOVA} نیز نام برده می‌شود
	\end{itemize}
	
		\subsubsection{توضیح \lr{Split-Plot ANOVA}}
	
	\lr{Split-Plot ANOVA} نام دیگر \lr{Mixed-Measures ANOVA} است که از کشاورزی نشات گرفته:
	
	\textbf{تاریخچه نام:}
	\begin{itemize}
		\item در آزمایش‌های کشاورزی، زمین‌ها (\lr{plots}) به قطعات کوچک‌تر (\lr{sub-plots}) تقسیم می‌شدند
		\item عامل اصلی (مثل نوع کود) به کل زمین اختصاص می‌یافت
		\item عامل فرعی (مثل نوع بذر) به قطعات کوچک اختصاص می‌یافت
	\end{itemize}
	
	\textbf{در مطالعه پارکینسون:}
	\begin{itemize}
		\item \textbf{\lr{Whole Plot Factor}:} گروه (نوروفیدبک یا کنترل) - بین‌آزمودنی
		\item \textbf{\lr{Sub-Plot Factor}:} زمان (پیش یا پس) - درون‌آزمودنی
		\item هر شرکت‌کننده فقط در یک گروه قرار دارد، اما در هر دو زمان اندازه‌گیری می‌شود
	\end{itemize}
	
	\textbf{مزایای این طرح:}
	\begin{itemize}
		\item کنترل بهتر واریانس فردی (درون‌آزمودنی)
		\item قدرت آماری بیشتر برای تشخیص تعامل گروه × زمان
		\item کاهش تعداد شرکت‌کنندگان مورد نیاز
	\end{itemize}
	
	\subsubsection{چرا از \lr{Two-Way ANOVA} استفاده نکردیم؟}
	
	\textbf{تفاوت کلیدی بین \lr{Mixed-Measures} و \lr{Two-Way ANOVA}:}
	
	\begin{table}[h]
		\centering
		\caption{مقایسه انواع تحلیل واریانس}
		\begin{tabular}{|p{4cm}|p{5cm}|p{5cm}|}
			\hline
			\textbf{ویژگی} & \textbf{\lr{Two-Way ANOVA}} & \textbf{\lr{Mixed-Measures ANOVA}} \\
			\hline
			ساختار داده & هر دو عامل بین‌آزمودنی & یک عامل بین‌آزمودنی + یک عامل درون‌آزمودنی \\
			\hline
			استقلال مشاهدات & همه مشاهدات مستقل & مشاهدات درون‌آزمودنی وابسته، بین‌آزمودنی مستقل \\
			\hline
			فرضیات & همگنی واریانس‌ها & همگنی واریانس‌ها + کرویت (sphericity) \\
			\hline
			خطای استاندارد & یک نوع خطا & دو نوع خطا (بین‌آزمودنی و درون‌آزمودنی) \\
			\hline
		\end{tabular}
	\end{table}
	
	\textbf{در مطالعه پارکینسون:}
	\begin{itemize}
		\item هر بیمار فقط در یک گروه است (بین‌آزمودنی)
		\item اما همان بیمار در دو زمان اندازه‌گیری می‌شود (درون‌آزمودنی)
		\item مشاهدات پیش و پس برای همان فرد وابسته هستند
		\item \lr{Two-Way ANOVA} فرض استقلال همه مشاهدات را نقض می‌کند
	\end{itemize}
	
	\textbf{مثال اشتباه اگر از \lr{Two-Way ANOVA} استفاده کنیم:}
	\begin{itemize}
		\item نادیده گرفتن وابستگی درون‌فردی
		\item تخمین اشتباه خطای استاندارد
		\item افزایش خطای نوع اول (\lr{False Positive})
		\item کاهش قدرت آماری
	\end{itemize}
	
	\subsection{نمودار جریان طراحی مطالعه}
	
	\begin{figure}[h]
		\centering
		\begin{tikzpicture}[
			box/.style={rectangle, draw, minimum width=3.5cm, minimum height=1.2cm, text centered, font=\footnotesize, align=center},
			arrow/.style={thick, -stealth}
			]
			
			% Initial recruitment
			\node[box] (recruit) at (0,8) {جذب بیماران پارکینسون \\ \lr{(n=30)}};
			
			% Randomization
			\node[box] (random) at (0,6.5) {تصادفی‌سازی};
			
			% Group 1 - NF
			\node[box] (nf1) at (-4,5) {گروه ۱: نوروفیدبک \\ \lr{(n=15)}};
			\node[box] (nf2) at (-4,3.5) {\rl{هفته ۱-۴: }\\ rt-fMRI نوروفیدبک \\ + \rl{تمرین تصویرسازی حرکتی}};
			\node[box] (nf3) at (-4,2) {\rl{هفته ۵-۱۰: }\\ \rl{ادامه تصویرسازی} \\ \rl{+ تمرین }Nintendo Wii};
			
			% Group 2 - MOT
			\node[box] (mot1) at (4,5) {گروه ۲: کنترل \\ \lr{(n=15)}};
			\node[box] (mot2) at (4,3.5) {\rl{هفته ۱-۴}: \\ \rl{تمرین حرکتی} \\ Nintendo Wii};
			\node[box] (mot3) at (4,2) {\rl{هفته ۵-۱۰:} \\ \rl{ادامه تمرین حرکتی} \\ Nintendo Wii};
			
			% Assessment points
			\node[box] (pre) at (0,0.5) {پیش‌آزمون: \\ MDS-UPDRS \\ "off-medication"};
			\node[box] (mid) at (0,-0.8) {\rl{ارزیابی میانی (هفته ۴)}: \\ MDS-UPDRS \\ "on-medication"};
			\node[box] (post) at (0,-2.1) {پس‌آزمون: \\ MDS-UPDRS \\ "off-medication"};
			
			% Arrows
			\draw[arrow] (recruit) -- (random);
			\draw[arrow] (random) -- (-2,5.5) -- (nf1);
			\draw[arrow] (random) -- (2,5.5) -- (mot1);
			\draw[arrow] (nf1) -- (nf2);
			\draw[arrow] (nf2) -- (nf3);
			\draw[arrow] (mot1) -- (mot2);
			\draw[arrow] (mot2) -- (mot3);
			\draw[arrow] (nf3) -- (-2,0.5) -- (pre);
			\draw[arrow] (mot3) -- (2,0.5) -- (pre);
			\draw[arrow] (pre) -- (mid);
			\draw[arrow] (mid) -- (post);
			
		\end{tikzpicture}
		\caption{نمودار جریان طراحی مطالعه با جزئیات مداخلات}
	\end{figure}
	
	\textbf{ویژگی‌های کلیدی طراحی:}
	\begin{itemize}
		\item \textbf{طرح ترکیبی:} عامل بین‌آزمودنی (گروه) + عامل درون‌آزمودنی (زمان)
		\item \textbf{کنترل‌شده تصادفی:} تخصیص تصادفی به گروه‌ها
		\item \textbf{اندازه‌گیری مکرر:} ارزیابی در سه نقطه زمانی
		\item \textbf{کورسازی:} ارزیاب کور نسبت به گروه‌بندی
	\end{itemize}
	
	\section{نتایج}
	
	\subsection{تعامل معنادار گروه × زمان}
	
	\begin{table}[h]
		\centering
		\caption{نتایج تحلیل واریانس ترکیبی}
		\begin{tabular}{lcccc}
			\toprule
			\textbf{گروه} & \textbf{پیش‌تمرین} & \textbf{پس‌تمرین} & \textbf{تغییر} & \textbf{فاصله اطمینان ۹۵٪} \\
			\midrule
			نوروفیدبک & - & - & \lr{-4.5} امتیاز & (\lr{-2.5} تا \lr{-6.6}) \\
			کنترل & - & - & \lr{-1.9} امتیاز & (\lr{+2.3} تا \lr{-6.8}) \\
			\bottomrule
		\end{tabular}
	\end{table}
	
	\textbf{تحلیل اثرات ساده:}
	\begin{itemize}
		\item گروه نوروفیدبک بهبود معناداری در نمرات MDS-UPDRS از پیش به پس‌تمرین نشان داد
		\item گروه کنترل بهبود کمتری داشت که از نظر آماری معنادار نبود
	\end{itemize}
	
	\subsection{تفسیر}
	
	اگرچه تعامل اصلی گروه × زمان در اندازه‌گیری اولیه معنادار نبود (\lr{p = 0.11})، اما:
	
	\begin{enumerate}
		\item بهبود \lr{4.5}امتیازی در گروه نوروفیدبک از نظر بالینی معنادار بود 
		\item برای برخی اندازه‌گیری‌های ثانویه، تعامل گروه × زمان معنادار بود
		\item این نتایج نشان می‌دهد که نوروفیدبک ممکن است فرآیند یادگیری مهارت‌های حرکتی را تسریع کند
	\end{enumerate}
	
	\section{نتیجه‌گیری}
	
	این مطالعه نمونه عالی از \textbf{طرح ترکیبی} است که در آن:
	\begin{itemize}
		\item دو گروه مستقل (نوروفیدبک و کنترل) در طول زمان پیگیری شدند
		\item هدف بررسی این بود که آیا یک گروه بیش از گروه دیگر در طول زمان تغییر کرده است
		\item از تحلیل واریانس اندازه‌گیری‌های مکرر ترکیبی استفاده شد
		\item تمرکز اصلی بر روی تعامل گروه × زمان بود
	\end{itemize}
	
	این روش آماری برای ارزیابی مداخلات درمانی در شرایط کنترل‌شده بسیار مفید است و امکان تعیین اثربخشی نسبی مداخلات مختلف را فراهم می‌کند.
	
	
	\subsection{روش‌های آماری استفاده شده}
	
	تحلیل داده‌ها با استفاده از نرم‌افزار SPSS نسخه ۲۰ انجام شد.
	
	\textbf{آزمون‌های مورد استفاده:}
	\begin{itemize}
		\item \textbf{Independent samples t-test:} برای بررسی تفاوت‌های پایه بین گروه‌ها
		\item \textbf{Paired samples t-test:} برای بررسی تغییرات درون‌گروهی از پیش تا پس مداخله
		\item \textbf{ANCOVA:} برای مقایسه نمرات پس‌آزمون با کنترل نمرات پیش‌آزمون
		\item \textbf{Repeated measures ANOVA:} برای بررسی تغییرات وابسته به زمان
		\item \textbf{FDR correction:} برای کنترل مقایسات چندگانه
	\end{itemize}
	
	\subsection{محدودیت‌های مطالعه}
	
	\begin{itemize}
		\item عدم وجود گروه کنترل با \lr{sham feedback} در اسکنر
		\item حجم نمونه کوچک (مطالعه \lr{Phase I})
		\item عدم پیگیری بلندمدت
		\item عدم تجویز استراتژی تصویرسازی مشخص
	\end{itemize}
	
	\section{مراجع}
	
	\begin{latin}
		Subramanian, L., Morris, M. B., Brosnan, M., Turner, D. L., Morris, H. R., \& Linden, D. E. (2016). Functional magnetic resonance imaging neurofeedback-guided motor imagery training and motor training for Parkinson's disease: randomized trial. \textit{Frontiers in Behavioral Neuroscience}, 10, 111.
		
		Available at: \url{https://www.frontiersin.org/journals/behavioral-neuroscience/articles/10.3389/fnbeh.2016.00111/full}
	\end{latin}
	
\end{document}
\documentclass[12pt]{article}
\usepackage{amsmath}
\usepackage{amsfonts}
\usepackage{amssymb}
\usepackage{graphicx}
\usepackage{tikz}
\usepackage{pgfplots}
\pgfplotsset{compat=1.17}
\usepackage{array}
\usepackage{booktabs}
\usepackage{hyperref}
\usepackage{geometry}
\geometry{margin=2.5cm}
\usepackage{xepersian}
\settextfont[Path="./", Extension=".ttf"]{XB-Niloofar}
\setdigitfont[Path="./", Extension=".ttf"]{XB-Niloofar}
\begin{document}
	
	\title{آزمون‌های آماری در تحقیقات EEG}
	\author{محمدمهدی شریف بیگی}
	\maketitle
	
	\section{انتخاب آزمون آماری برای تحقیقات EEG}
	
	\subsection{آزمون t مستقل (\lr{Independent Samples t-test})}
	
	این آزمون برای مقایسه میانگین دو گروه مستقل و غیرمرتبط استفاده می‌شود.
	
	\textbf{مثال اول:} آیا میانگین قدرت موج آلفا در بیماران افسردگی با افراد سالم متفاوت است؟
	
	\textbf{چرا این مثال مناسب است؟}
	این مثال دارای ویژگی‌های زیر است:
	\begin{itemize}
		\item دو گروه مستقل: بیماران افسردگی و افراد سالم (هر فرد فقط در یک گروه قرار دارد)
		\item یک متغیر وابسته پیوسته: قدرت موج آلفا (µV²)
		\item اندازه‌گیری یک‌بار: هر آزمودنی فقط یک نمره دارد
		\item گروه‌ها مستقل: انتخاب یک آزمودنی تأثیری بر انتخاب دیگری ندارد
	\end{itemize}
	
	\textbf{مثال دوم:} آیا دامنه P300 در کودکان اوتیسم با کودکان عادی تفاوت دارد؟
	
	\textbf{چرا این مثال مناسب است؟}
	\begin{itemize}
		\item دو گروه طبیعی و جداگانه: کودکان اوتیسم و کودکان عادی
		\item متغیر وابسته کمّی: دامنه P300 (میکروولت)
		\item طراحی بین‌آزمودنی: هر کودک فقط در یک گروه
	\end{itemize}
	
	\textbf{مثال ساده:} آیا میانگین قد دانش‌آموزان دختر با پسر متفاوت است؟
	
	\textbf{چرا این مثال ساده مناسب است؟}
	\begin{itemize}
		\item دو گروه مستقل و طبیعی: دختر و پسر
		\item یک متغیر پیوسته: قد (سانتی‌متر)
		\item هر فرد فقط در یک گروه قرار دارد
	\end{itemize}
	
	\subsection{آزمون ANOVA یک‌طرفه (\lr{One-Way ANOVA})}
	
	برای مقایسه میانگین سه گروه یا بیشتر که مستقل هستند استفاده می‌شود.
	
	\textbf{مثال اول:} آیا قدرت موج بتا در بیماران پارکینسون، لرزش ضروری و افراد سالم متفاوت است؟
	
	\textbf{چرا این مثال مناسب است؟}
	\begin{itemize}
		\item سه گروه مستقل: پارکینسون، لرزش ضروری، سالم
		\item یک فاکتور (عامل): وضعیت بیماری با سه سطح
		\item متغیر وابسته پیوسته: قدرت موج بتا
		\item نیاز به آزمون پس‌آزمون برای مشخص کردن کدام جفت گروه‌ها تفاوت دارند
	\end{itemize}
	
	\textbf{مثال دوم:} آیا دامنه N400 در سه گروه سنی (جوان، میان‌سال، سالمند) متفاوت است؟
	
	\textbf{چرا این مثال مناسب است؟}
	\begin{itemize}
		\item سه گروه سنی مختلف و مستقل
		\item یک متغیر مستقل طبقه‌بندی: گروه سنی
		\item هدف: مقایسه تأثیر سن بر پردازش معنایی (\lr{N400})
	\end{itemize}
	
	\textbf{مثال ساده:} آیا نمره امتحان ریاضی در سه کلاس مختلف (\lr{A, B, C}) متفاوت است؟
	
	\textbf{چرا این مثال ساده مناسب است؟}
	\begin{itemize}
		\item سه گروه مستقل: سه کلاس مختلف
		\item یک متغیر مستقل: کلاس (با سه سطح)
		\item متغیر وابسته پیوسته: نمره امتحان
		\item نیاز به آزمون پس‌آزمون برای مشخص کردن کدام کلاس‌ها تفاوت دارند
	\end{itemize}
	
	\subsection{آزمون t زوجی (\lr{Paired Samples t-test})}
	
	برای مقایسه یک گروه در دو زمان مختلف استفاده می‌شود.
	
	\textbf{مثال اول:} آیا قدرت موج تتا قبل و بعد از مداخله مراقبه تغییر می‌کند؟
	
	\textbf{چرا این مثال مناسب است؟}
	\begin{itemize}
		\item همان آزمودنی‌ها در دو زمان: قبل و بعد از مداخله
		\item طراحی درون‌آزمودنی: هر فرد کنترل خودش است
		\item دو اندازه‌گیری وابسته: نمرات قبل و بعد همبسته هستند
		\item هدف: بررسی تأثیر مداخله مراقبه
	\end{itemize}
	
	\textbf{مثال دوم:} آیا دامنه P300 قبل و بعد از درمان دارویی متفاوت است؟
	
	\textbf{چرا این مثال مناسب است؟}
	\begin{itemize}
		\item طراحی قبل-بعد: همان بیماران در دو نقطه زمانی
		\item کنترل متغیرهای مزاحم: تفاوت‌های فردی کنترل می‌شوند
		\item ارزیابی اثربخشی درمان دارویی
	\end{itemize}
	
	\textbf{مثال ساده:} آیا وزن دانشجویان قبل و بعد از ترم تغییر کرده است؟
	
	\textbf{چرا این مثال ساده مناسب است؟}
	\begin{itemize}
		\item همان افراد در دو زمان مختلف: ابتدا و انتهای ترم
		\item هر فرد کنترل خودش است: تفاوت‌های فردی حذف می‌شوند
		\item دو اندازه‌گیری از یک متغیر پیوسته: وزن
	\end{itemize}
	
	\subsection{آزمون ANOVA اندازه‌گیری مکرر (\lr{Repeated Measures ANOVA})}
	
	برای مقایسه یک گروه در سه یا چند زمان/شرایط مختلف استفاده می‌شود.
	
	\textbf{مثال اول:} آیا دامنه N400 در سه نوع جمله (هم‌خوان، ناهم‌خوان، بی‌معنی) متفاوت است؟
	
	\textbf{چرا این مثال مناسب است؟}
	\begin{itemize}
		\item همان آزمودنی‌ها سه شرایط مختلف را تجربه می‌کنند
		\item طراحی درون‌آزمودنی: هر فرد در هر سه شرایط شرکت می‌کند
		\item سه سطح از یک فاکتور: نوع جمله
		\item کنترل بهتر تفاوت‌های فردی
	\end{itemize}
	
	\textbf{مثال دوم:} آیا قدرت موج گاما در سه وضعیت (استراحت، تمرکز، خواب) تغییر می‌کند؟
	
	\textbf{چرا این مثال مناسب است؟}
	\begin{itemize}
		\item سه وضعیت مختلف ذهنی که همه افراد آن را تجربه می‌کنند
		\item امکان مقایسه الگوهای فعالیت مغزی در شرایط مختلف
		\item نیاز به آزمون پس‌آزمون برای مشخص کردن تفاوت‌های جفتی
	\end{itemize}
	
	\subsection{آزمون ANOVA طرح ترکیبی (\lr{Mixed-Design ANOVA})}
	
	برای بررسی تأثیر دو فاکتور که یکی بین‌گروهی و دیگری درون‌گروهی است.
	
	\textbf{مثال اول:} آیا تغییر در قدرت موج SMR در طول نوروفیدبک در گروه واقعی نسبت به گروه ساختگی متفاوت است؟
	
	\textbf{چرا این مثال مناسب است؟}
	\begin{itemize}
		\item فاکتور بین‌گروهی: نوع درمان (واقعی vs ساختگی)
		\item فاکتور درون‌گروهی: زمان (قبل vs بعد)
		\item بررسی تعامل: آیا اثر زمان بسته به نوع درمان متفاوت است؟
		\item طراحی کنترل‌دار برای ارزیابی اثربخشی درمان
	\end{itemize}
	
	\textbf{مثال دوم:} آیا تأثیر داروی ضدافسردگی بر دامنه P300 در مردان و زنان متفاوت است؟
	
	\textbf{چرا این مثال مناسب است؟}
	\begin{itemize}
		\item فاکتور بین‌گروهی: جنسیت (مرد vs زن)
		\item فاکتور درون‌گروهی: زمان درمان (قبل vs بعد)
		\item سوال اصلی: آیا پاسخ به درمان در دو جنس متفاوت است؟
		\item بررسی اثر متقابل جنسیت و درمان
	\end{itemize}
	
	\subsection{آزمون همبستگی (\lr{Correlation})}
	
	برای بررسی رابطه خطی بین دو متغیر پیوسته استفاده می‌شود.
	
	\textbf{مثال اول:} آیا رابطه‌ای بین دامنه P300 و نمره حافظه کاری وجود دارد؟
	
	\textbf{چرا این مثال مناسب است؟}
	\begin{itemize}
		\item دو متغیر پیوسته: دامنه P300 (میکروولت) و نمره حافظه کاری (درصد)
		\item هدف: بررسی رابطه بین فعالیت عصبی و عملکرد شناختی
		\item بدون دستکاری متغیر: مطالعه رابطه طبیعی بین متغیرها
		\item امکان پیش‌بینی: نمره حافظه بر اساس P300
	\end{itemize}
	
	\textbf{مثال دوم:} آیا ارتباطی بین قدرت موج آلفا و نمره اضطراب وجود دارد؟
	
	\textbf{چرا این مثال مناسب است؟}
	\begin{itemize}
		\item دو متغیر کمّی: قدرت آلفا و نمره اضطراب
		\item مطالعه رابطه بین فیزیولوژی مغز و وضعیت روانی
		\item بدون گروه‌بندی: همه افراد در یک طیف پیوسته
		\item کاربرد بالینی: شناسایی نشانگرهای عصبی اضطراب
	\end{itemize}
	
	\section{نکات مهم در انتخاب آزمون}
	
	\begin{enumerate}
		\item \textbf{تعداد گروه‌ها:} دو گروه = t-test، بیش از دو گروه = ANOVA
		\item \textbf{استقلال گروه‌ها:} گروه‌های مستقل = آزمون‌های بین‌گروهی، گروه‌های وابسته = آزمون‌های درون‌گروهی
		\item \textbf{نوع متغیر:} متغیرهای طبقه‌ای = مقایسه میانگین، متغیرهای پیوسته = همبستگی
		\item \textbf{طراحی مطالعه:} ترکیب فاکتورهای بین‌گروهی و درون‌گروهی = طرح ترکیبی
	\end{enumerate}
	
\end{document}